\chapter{Конструкторский раздел}
\label{cha:design}

\section{Правила реализуемой игры}

Два игрока должны зайти на сервер, только в этом случае игра начинается в режиме друг против друга.
Игра длится до того момента, пока один из пользователей не закроет вкладку браузера с игрой. Сама игра состоит из следующих пунктов:

\begin{itemize}
	\item Пользователи могут управлять только своей ракеткой.
	\item Движение осуществляется по оси y.
	\item Игра состоит из раундов:
	\begin{itemize}
		\item Цель раунда: забить мяч в ворота противника, при этом не пропустив его в свои.
		\item В начале каждого раунда мяч начинает движение в сторону победившего в прошлом раунде игрока.
	\end{itemize}
	\item Мяч двигается по линейной траектории, при столкновении с препятствием его траектория меняется в соответствии с углом столкновения.
	\item При пропуске мяча игроком очко записывается противнику, раунд начинается заново.
\end{itemize}

\section{Состав, структура и взаимодействие компонентов}

Для разработки приложения выбрана централизованная клиент-серверная архитектура с одним сервером, обслуживающим клиентов.

\subsection{Серверная часть}

На сервере выполняются:

\begin{itemize}
	\item Создание новой игровых сессий.
	\item Подключение пользователя к существующей игровой сессии.
	\item Управление состояниями игроков (победа/поражение; оповещение о том, что противник покинул игру; оповещение о том, что противник присоединился к игре).
	\item Управление игровым процессом (например определение того, что мяч вышел за пределы чьих то ворот).
	\item Управление мячом.
\end{itemize}

\subsection{Клиентская часть}

Клиентское приложение содержит графический интерфейс игры и выполняет следующие функции:

\begin{itemize}
	\item Отправка данных о перемещении пользовательской ракетки.
	\item Отрисовка действий противника.
	\item Отрисовка перемещения мяча.
	\item Отображение игровой статистики: счет, состояние игры.
	\item Оправка на сервер команд о запуске игры или установке ее на паузу.
\end{itemize}

Для передачи данных используется socket.io.

\section{Алгоритм работы сервера}
Для взаимодействия клиентского и серверного приложения используются следующие команды:

\begin{itemize}
	\item Запуск игры.
	\item Остановка игры.
	\item Ожидание игроком оппонента.
	\item Оповещение о подборе оппонента.
	\item Оповещение о выходе оппонента.
	\item Проигрыш.
	\item Победа.
	\item Обновление состояния игры.
\end{itemize}

\section{Синхронизация между клиентом и сервером}

Синхронизация между клиентом и сервером происходит следующим образом:

\begin{itemize}
	\item Клиент отправляет на сервер информацию о положении ракетки при ее перемещении.
	\item Сервер периодически отправляет клиенту информацию о состоянии игры.
\end{itemize}

Информация, отправляемая клиенту сервером хранит следующие данные:

\begin{itemize}
	\item Информация о мяче: текущие координаты, вектор движения.
	\item Позиция оппонента.
\end{itemize}

Отправка сообщений происходит по таймеру, время которого задается через конфигурацию сервера.

Для передачи данных была выбрана библиотека socket.io, так как это единственный известный мне вариант поддержки WebSockets на сервере. Socket.io не идеальный вариант, потому что она несколько избыточна, так как поддерживает передачу данных не только через WebSockets, но так же через flash сокеты и еще несколькими способами. Но игра идет с адекватной скоростью только при использовании WebSockets. Так же в планах перейти на защищенное соединение, что позволит устанавливать соединения через некоторые прокси, которые по каким-то причинам не хотят передавать WebSocket трафик.
%%% Local Variables:
%%% mode: latex
%%% TeX-master: "rpz"
%%% End:
