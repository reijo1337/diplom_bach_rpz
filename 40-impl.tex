\chapter{Технологический раздел}
\label{cha:impl}

В данном разделе описываются средства, используемые для разработки программного продукта, требования для функционирования ПО, описываются результаты тестирования программного продукта.

\section{Выбор средств разработки}

\subsection{Выбор языка программирования}

Для написания программного продукта используется язык Python, так как для него существует множество готовых решений для работы с нейронными сетями. Так же выбранный язык сочетает в себе возможности функционального, структурного и объектно-ориентированного подходов, что позволяет кратко описывать математические конструкции, необходимые для решения поставленной задачи. Так же для данного языка существует большое количество различных математических библиотек, упрощающие работу с комплексными числами, и библиотек для обработки цифровых сигналов.

\subsection{Выбор среды программирования и отладки}

В качестве среды разработки для языка Python была выбрана кроссплатформенная IDE PyCharm. Предоставляет средства для анализа кода, графический отладчик, инструмент для запуска юнит-тестов. На данный момент PyCharm является бесплатным для образовательных учреждений и проектов с открытым исходным кодом.

Выбор данной среды разработки обусловлен следующими предоставляемыми возможностями, упрощающими разработку приложения и способствующими повышению качества исходного кода:

\begin{itemize}
	\item статический анализ кода;
	\item встроенный отладчик;
	\item навигация по проекту и исходному коду;
	\item рефакторинг;
	\item поддержка систем контроля версий.
\end{itemize}

\subsection{Используемые библиотеки}

Для упрощения реализации математических операций используется библиотека NumPy. Для считывания и записи аудио-файлов, а так же обработки сигналов используется библиотека LibROSA. Для работы с нейронной сетью используется библиотека TensorFlow.

\section{Система контроля версий}

В процессе разработки программы использовалась система контроля версий
Git. Система контроля версий позволяет вносить в проект атомарные изменения, направленные на решения каких-либо задач. В случае обнаружения ошибок или из- менения требований, вне-сенные изменения можно отменить. Кроме того, с помощью системы контроля версий решается вопрос резервного копирования.

Особенности Git:

\begin{itemize}
	\item данная система контроля версий является децентрализованной, что позволяет иметь несколько независимых резервных копий проекта;
	\item поддерживается хостингом репозиториев GitLab;
	\item поддерживается средой разработки PyCharm;
	\item предоставляет широкие возможности для управления изменениями проекта и просмотра истории изменений.
\end{itemize}

\section{Требования к вычислительной системе}

Для запуска программы необходимо иметь установленный на ЭВМ интерпретатор для Python 2.7 с установленными библиотеками. 

Так как выбранный язык программирования является кроссплатформенным, то требований к использованию операционной системы нет.

Алгоритмы работают с большими объёмами комплексных данных, поэтому объём оперативной памяти компьютера не должен быть меньше 1 ГБ, желательна архитектура x64 (x86-64).

\section{Формат данных}

Входом и выходом программного продукта являются WAV-файлы. Формат WAVE имеет четкую структуру, описанную в \cite{wavv}.

WAV -- формат файла-контейнера для хранения записи оцифрованного аудиопотока, в котором для кодирования амплитуды вы- деляется фиксированное число бит.

WAV-файл состоит из двух частей. Одна из них -- заголовок файла, другая -- область данных. В заголовке файла хранится информация о:

\begin{itemize}
	\item размере файла;
	\item количестве каналов;
	\item частоте дискретизации;
	\item количестве бит в сэмпле (глубине звучания).
\end{itemize}

Длина заголовка составляет 44 байта. Область данных представляет собой набор амплитуд. 

В программе используется информация о количестве каналов и частоте дискретизации, а также амплитуды.

\section{Установка программного обеспечения}

%%% Local Variables:
%%% mode: latex
%%% TeX-master: "rpz"
%%% End:
