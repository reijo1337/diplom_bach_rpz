\Introduction

Информационные технологии в настоящее время имеют всё большее влияние в различных науках. Так за последние двадцать лет развитие получило такое направление как получение информации о музыке (англ. music information retrieval, MIR)\cite{MIR}. Это направление включает в себя анализ музыкальных произведений и связанных с ними метаданных с помощью вычислительных методов и объединяет в себе идеи музыковедения, цифровой обработки сигналов и
машинного обучения. К задачам MIR относятся определение жанра, транскрибирование, построение рекомендательной системы, определение настроения музыкального произведения, выделение источников, определение музыкальных инструментов и другие. На данный момент каждая задача является нетривиальной и не решена полностью\cite{Downie}.

Выделение источников является важной задачей, решение которой найдет применение во многих реальных задачах. Например, точность систем автоматизированного распознавания голоса (англ. automatic speech recognition, ASR) может быть увеличена с помощью выделения шума из аудио сигнала\cite{Maas}. Точность распознавания аккордов и оценки высоты тона можно улучшить, отделив вокал от музыки\cite{[Huang}. Автоматизацию создания нотных партитур для музыкальных произведений можно провести с помощью методов автоматического транскрибирования к выделенным из исходного сигнала сигналы источников.

Методы выделения источников подразумевает определение и воспроизведение аудио сигналов, являющихся компонентами исходного сигнала.

Полное выделение источников из музыкального произведения включает в себя задачу классификации использованных при записи источников, их определение и воспроизведение. Но на данный момент решение задачи полного выделения источников из полифонического музыкального произведения считается крайне сложным или невозможным \cite{Hsu}. По этому целью систем автоматического выделения аудио источников обычно является выделение заранее известных источников, например вокал, бас, барабаны.

Целью даной работы является разработка метода выделения голосовой составляющей из монофонического аудио сигнала. Для решения поставленной задачи необходимо:

\begin{itemize}
	\item проанализировать предметную область;
	\item проанализировать существующие методы выделения источников из аудио сигнала;
	\item на основе полученных во время анализа данных разработать собственный метод выделения голосовой составляющей из монофонического аудио сигнала;
	\item реализовать разработанный метод в программном продукте.
\end{itemize}