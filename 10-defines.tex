\Defines % Необходимые определения. Вряд ли понадобться
\begin{description}
\item[Дискретизация] преобразование непрерывного информационного множества аналоговых сигналов в дискретное множесто.
\item[Частота] физическая величина, характеристика периодического процесса, равна количеству повторений или возникновения событий в единицу времени. Единица измерения -- герцы (Гц).
\item[Аккомпанемент] сопровождение одним или несколькими инструментами, а также оркестром сольной партии (певца, инструменталиста, хора и других). Сопроводителя называют аккомпаниатором. Аккомпанементом также называют гармоническое и ритмическое сопровождение основной мелодии, голоса.
\item[Музыкальное произведение] всякая музыкальная пьеса, в том числе, народная песня или инструментальная импровизация. Также музыкальное произведение есть категория музыкальной эстетики, обозначающая ограниченный историческими и культурными рамками результат композиторской деятельности. Музыкальным произведениям свойственны внутренняя завершённость и мотивированность целого, индивидуализированность содержания и формы, за которыми стоит личность автора, детальная фиксация нотной (или другого типа) записи, предполагающая искусство исполнительской интерпретации.
\item[Звук] физическое явление, представляющее собой распространение в виде упругих волн механических колебаний в твёрдой, жидкой или газообразной среде.
\item[Вокал (или пение)] исполнениt мелодии с помощью голоса человека.
\item[Тембр] (обертоновая) окраска звука, <<качество тона>>.
\end{description}

%%% Local Variables:
%%% mode: latex
%%% TeX-master: "rpz"
%%% End:
