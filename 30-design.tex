\chapter{Конструкторский раздел}
\label{cha:design}

\section{Архитектура программного продукта Разрабатываемый}

Разрабатываемый программный продукт состоит из следующих частей:

\begin{enumerate}
	\item модуль обработки входных данных;
	\item модуль генерации частотно-временной маски для вокала;
	\item модуль применения маски к исходному сигалу;
	\item модуль формирования выходных данных.
\end{enumerate}


\subsubsection{ДОПИСАТЬ}

\section{Алгоритмы}

\subsection{Обработка входных данных}

Чтение аудио-файлов происходит с помощью библиотеки. Результатом работы библиотечной функции является информация о частоте дискретизации и структура со значениями амплитуд сигнала. 

После считывания аудио-файла происходит построение его спектра при помощи оконного преобразования Фурье. Результатом работы этого алгоритма является структура, в которой хранятся на- боры частот по отрезкам времени.

\subsection{Генерация частотно-временных масок}

Для получения частотно-временных масок используется рекурентная нейронная сеть. На ее вход подается результат применения ОПФ к исходному сигналу. На выходе сеть генерирует две частотно-временных маски, для аккомпанимента и вокала соответсвенно.

\subsection{Описание сети}

z
%%% Local Variables:
%%% mode: latex
%%% TeX-master: "rpz"
%%% End:
